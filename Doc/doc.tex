\documentclass[a4paper,titlepage,openany,12pt]{article}

% BASICS
\usepackage[english]{babel}
\usepackage[utf8]{inputenc} % Reconnaissance des accents en entrée
\usepackage[T1]{fontenc} % Reconnaissance des accents en sortie
%\usepackage{lmodern} % Ajout de polices
\usepackage{stmaryrd} % Ajout de symboles

\usepackage{indentfirst} % Alinéa du premier paragraphe
\usepackage{graphicx} % Pour utiliser des graphiques
\usepackage{array} % Pour utiliser des tableaux
\usepackage{amsmath, amsfonts, amssymb, }%dsfont} % Pour les formules mathématiques et polices associées et symboles associés et fonction indicatrice
\usepackage[mathscr]{eucal} % Pour la police mathcal

% AVANCES

\usepackage{dirtree}

\usepackage[svgnames]{xcolor} % Couleurs
\usepackage{hyperref} % Hyperrefs sur le PDF
\hypersetup{colorlinks=true,linkcolor=blue} % ... et en bleus

\usepackage{pdfpages} % Ajout de documents PDF dans le rapport
\usepackage{pdflscape} % Utilisation du format paysage
\usepackage{todonotes} % Explicite
\usepackage{glossaries} % Pour créer un glossaire : http://blog.dorian-depriester.fr/latex/utilisation-du-package-glossaries
\usepackage{pdflscape} % Pour passer certaines pages en orientation portrait
\usepackage{rotating} % Pour passer certaines pages en orientation portrait
\usepackage{chngcntr} % Pour modifier la numérotation des graphiques : https://texblog.org/2014/12/04/continuous-figuretable-numbering-in-latex/
\usepackage{multirow} % Pour diviser une case d'un tableau

\usepackage{enumitem} % Pour jouer sur les paramètres des itemize

% MISE EN PAGE
\usepackage[left=25mm,right=25mm,top=25mm,bottom=25mm]{geometry}
\usepackage{lastpage} % numérotation par rapport à la dernière page
\usepackage{here} % Pour bloquer les éléments avec [H]

\usepackage{fancyhdr} % Agencement des en-tête et pied-de-page
\pagestyle{fancy}
\fancyhf{}
\fancyhead[L]{\nouppercase \leftmark}
\fancyfoot[R]{\thepage}
\fancyfoot[L]{Projet SDTD - Pile SMACK - Documentation}
\renewcommand{\footrulewidth}{1pt}


\fancypagestyle{lscape}{% 
\fancyhf{} % clear all header and footer fields 
\fancyfoot[LE]{%
\begin{textblock}{20}(1,5){\rotatebox{90}{\leftmark}}\end{textblock}
\begin{textblock}{1}(13,10.5){\rotatebox{90}{\thepage}}\end{textblock}}
\fancyfoot[LO] {%
\begin{textblock}{1}(13,10.5){\rotatebox{90}{\thepage}}\end{textblock}
\begin{textblock}{20}(1,13.25){\rotatebox{90}{\rightmark}}\end{textblock}}
\renewcommand{\headrulewidth}{0pt} 
\renewcommand{\footrulewidth}{0pt}}


\usepackage{titletoc} % Agencement de la Table des matières
%\dottedcontents{part}[3.5em]{\Large \bf \addvspace{10pt}}{2.2em}{0em}
%\dottedcontents{chapter}[2.3em]{\large \bf \addvspace{10pt}}{1.5em}{0em}
\dottedcontents{section}[5em]{\bf \addvspace{5pt}}{2.2em}{1.2em}
\dottedcontents{subsection}[8em]{\addvspace{3.5pt}}{3em}{.7em}

\usepackage{titlesec} % Agencement des Titres
%\titlespacing{\chapter}{0pt}{*3}{*9}
%\titleformat{\chapter}[display]
%{\normalfont\Large\filcenter \bfseries}
%{\rule[1mm]{15mm}{1mm} \hspace{4mm} \hspace{5mm} \rule[1mm]{15mm}{1mm}}
%{1pc}
%{%\titlerule
%\vspace{1pc}%
%\Huge}

\titlespacing*{\subsection} {15pt}{3.25ex plus 1ex minus .2ex}{1.5ex plus .2ex}

% Affichage Tableaux et figures
\usepackage{caption}

	% Affichage figure en couleur
\renewcommand{\figurename}{FIGURE}
\captionsetup[figure]{labelfont={color=DarkRed}}
	% Affichage table -> tableau (et petites capitales)
%\renewcommand\tablename{\textsc{Tableau}}
	% Affichage tableau en couleur 
%\renewcommand{\tablename}{TABLE}
%\captionsetup[table]{labelfont={color=DarkRed}}

%%%%%%%%%%%%%%%%%%%%%%%%%%%%%%%%%%%%%%%%%%%%%%%%%%%%%%%%%%%%
\begin{document}

%\frontmatter

\includepdf{PageDeGarde.pdf}


%%%%%%%%%%%%%%%%%%%%%%%%%%%%%%%%%%%%%%%%%%%%%%%%%%%%%%%%%%%%
\tableofcontents
\newpage
%\listoffigures
%\listoftables
%\mainmatter
%%%%%%%%%%%%%%%%%%%%%%%%%%%%%%%%%%%%%%%%%%%%%%%%%%%%%%%%%%%%




%%%%%%%%%%%%%%%%%%%%%%%%%%%%%% CHAPTER 1 %%%%%%%%%%%%%%%%%%%%%%%%%%%%%%


\section{Compromission}
\vspace{-0.6cm}
\noindent\rule{\textwidth}{0.4pt}
\vspace{0.1cm}

\begin{figure}[H] %[H] : insertion à l'emplacement dans le texte

Merci de prendre connaissance de la faille CVE-2017-1000366 rencontrée dans nos systèmes et d'agir suivant les recommandations du département de sécurité informatique. \\

  \dirtree{%
          .1 \textbf{CVE-2017-1000366}.
          .2 \textbf{Type}.
          .3 Local exploit. 
          .3 Stack smashing.
          .2 \textbf{Service compromis}.
          .3 Bibliothèque standard C, GNU C Librairy (glibc). 
          .2 \textbf{Systèmes affectés}.
          .3 Linux, architecture x86.
          .3 OS : CentOS, Ubuntu, RedHat, Suse, Ubuntu, Fedora.
          .2 \textbf{Compromission}.
          .3 Confidentialité [Totalement compromis].
          .3 Intégrité [Totalement compromis].
          .3 Authentifications [outrepassées]\\.
        }
        
	%\caption{Note du département de sécurité informatique sur la faille indexée CVE-2017-1000366}

\end{figure}

\section{todo}





\section*{Bibliographie}
\addcontentsline{toc}{section}{Bibliographie}
\vspace{-0.6cm}
\noindent\rule{\textwidth}{0.4pt}

\begin{itemize}
\item \url {http://www.cvedetails.com/cve/CVE-2017-1000366/}
\item \url{https://www.exploit-db.com/exploits/42275/}
\item \url{https://www.suse.com/support/kb/doc/?id=7020973}
\item \url{https://securitytracker.com/id/1038712}
\item \url{https://www.qualys.com/2017/06/19/stack-clash/stack-clash.txt}
\end{itemize}

%%%%%%%%%%%%%%%%%%%%%%%%%%%%%% ANNEXES %%%%%%%%%%%%%%%%%%%%%%%%%%%%%%%%

\appendix

  %\chapter{Login}
  %\label{appendix:login}
  %\includegraphics[width=\textwidth]{login2.PNG}
 

\end{document}
\grid
